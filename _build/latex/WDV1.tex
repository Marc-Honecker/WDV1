%% Generated by Sphinx.
\def\sphinxdocclass{jupyterBook}
\documentclass[letterpaper,10pt,english]{jupyterBook}
\ifdefined\pdfpxdimen
   \let\sphinxpxdimen\pdfpxdimen\else\newdimen\sphinxpxdimen
\fi \sphinxpxdimen=.75bp\relax
\ifdefined\pdfimageresolution
    \pdfimageresolution= \numexpr \dimexpr1in\relax/\sphinxpxdimen\relax
\fi
%% let collapsible pdf bookmarks panel have high depth per default
\PassOptionsToPackage{bookmarksdepth=5}{hyperref}
%% turn off hyperref patch of \index as sphinx.xdy xindy module takes care of
%% suitable \hyperpage mark-up, working around hyperref-xindy incompatibility
\PassOptionsToPackage{hyperindex=false}{hyperref}
%% memoir class requires extra handling
\makeatletter\@ifclassloaded{memoir}
{\ifdefined\memhyperindexfalse\memhyperindexfalse\fi}{}\makeatother

\PassOptionsToPackage{warn}{textcomp}

\catcode`^^^^00a0\active\protected\def^^^^00a0{\leavevmode\nobreak\ }
\usepackage{cmap}
\usepackage{fontspec}
\defaultfontfeatures[\rmfamily,\sffamily,\ttfamily]{}
\usepackage{amsmath,amssymb,amstext}
\usepackage{polyglossia}
\setmainlanguage{english}



\setmainfont{FreeSerif}[
  Extension      = .otf,
  UprightFont    = *,
  ItalicFont     = *Italic,
  BoldFont       = *Bold,
  BoldItalicFont = *BoldItalic
]
\setsansfont{FreeSans}[
  Extension      = .otf,
  UprightFont    = *,
  ItalicFont     = *Oblique,
  BoldFont       = *Bold,
  BoldItalicFont = *BoldOblique,
]
\setmonofont{FreeMono}[
  Extension      = .otf,
  UprightFont    = *,
  ItalicFont     = *Oblique,
  BoldFont       = *Bold,
  BoldItalicFont = *BoldOblique,
]



\usepackage[Bjarne]{fncychap}
\usepackage[,numfigreset=1,mathnumfig]{sphinx}

\fvset{fontsize=\small}
\usepackage{geometry}


% Include hyperref last.
\usepackage{hyperref}
% Fix anchor placement for figures with captions.
\usepackage{hypcap}% it must be loaded after hyperref.
% Set up styles of URL: it should be placed after hyperref.
\urlstyle{same}


\usepackage{sphinxmessages}



        % Start of preamble defined in sphinx-jupyterbook-latex %
         \usepackage[Latin,Greek]{ucharclasses}
        \usepackage{unicode-math}
        % fixing title of the toc
        \addto\captionsenglish{\renewcommand{\contentsname}{Contents}}
        \hypersetup{
            pdfencoding=auto,
            psdextra
        }
        % End of preamble defined in sphinx-jupyterbook-latex %
        

\title{WDV1}
\date{Apr 20, 2023}
\release{}
\author{Chrsitian Müller, Marc Johannes Honecker und Martin H.\@{} Müser}
\newcommand{\sphinxlogo}{\vbox{}}
\renewcommand{\releasename}{}
\makeindex
\begin{document}

\pagestyle{empty}
\sphinxmaketitle
\pagestyle{plain}
\sphinxtableofcontents
\pagestyle{normal}
\phantomsection\label{\detokenize{intro::doc}}


\sphinxAtStartPar
In diesem Kurs soll Ihnen beigebracht werden, Computer zur Unterstützung diverser anderer Fächer zu verwenden. Als die dazu beste
Programmiersprache hat sich (momentan) Python herauskristallisiert, auf das sich dieser Kurs fokussiert. Python ist bereits seit
einigen Jahren State\sphinxhyphen{}Of\sphinxhyphen{}The\sphinxhyphen{}Art für viele Bereiche, sei es Scientific Computing oder (vor allem in den letzten Jahren) KI.
Deshalb ist im Moment ein Ende seines Siegeszuges noch nicht absehbar.

\sphinxAtStartPar
Für den Kurs haben wir einen “ganzheitlichen” Ansatz gewählt, der von üblichen Programmierkursen abweicht. Sie lernen also im
übertragenen Sinne zunächst ganze Sätze ohne mit vielen grammatikalischen Begriffen konfrontiert zu werden. Damit sollen Sie
schnell in die Lage versetzt werden kontinuierlich durch das Skript an Funktionen verschiedener Bibliotheken herangefürht zu
werden. Damit werden an einigen Beispielen auch Hintergrund z.B. zu Numerik oder Statistik formlos gelehrt oder es wird aufgezeigt,
wie die Probleme mit herkömmlichen Programmen (üblicherweise in Python Syntax) gelöst würden. Am Ende von WDV\sphinxhyphen{}1 sollten Sie dann in
der Lage sein, echte Python Bücher (quer) zu lesen, aber auch schnell andere Programmiersprachen zu erlernen, wie z.B. C++.

\sphinxAtStartPar
In den Hausaufgaben wird es zwei Testate geben, deren Bestehen Ihnen erlaubt an der finalen Leistungkontrolle, der Klausur,
teilzunehmen. Weder bei Testaten, noch Klausuren sind Computer erlaubt. Das klingt paradox, aber es wäre sonst zu einfach zu
mogeln. Leistungkontrollen werden vermutlich aus zwei, manchmal drei Teilen bestehen:
\begin{enumerate}
\sphinxsetlistlabels{\arabic}{enumi}{enumii}{}{.}%
\item {} 
\sphinxAtStartPar
Ein Teil, in dem Ihnen der Code gegeben wird und Sie vorhersagen müssen, was bei dei dessen Ausführung passiert.

\item {} 
\sphinxAtStartPar
Ein Teil, in dem Sie selbst Python Code (oder Pseudo\sphinxhyphen{}Code) schreiben.

\item {} 
\sphinxAtStartPar
Unter Umständen werden Sie noch aufgefordert, Fehler in einem gegebenen Code zu markieren/verbessern.

\end{enumerate}

\sphinxAtStartPar
\sphinxstylestrong{Dieses Skript ist kein Lehrbuch.} Details zur Syntax und andere Regeln werden in der Vorlesung und den Übungen erklärt oder
können in den angegebenen Referenzen nachgeschlagen werden. Dieses Skript ist eher eine Ansammlung von kleinen kommentierten
Code\sphinxhyphen{}Schnipseln.

\begin{DUlineblock}{0em}
\item[] \sphinxstylestrong{\Large Ablauf einer Lehreinheit}
\end{DUlineblock}

\sphinxAtStartPar
Eine Lehreinheit besteht aus 3 Lehrstunden zu je 45 Minuten. In der 1. Stunde werden die Hausaufgaben besprochen, in der 2. Stunde
neue Inhalte vermittelt und die 3. Stunde dient dazu, dass Studierende mit der Lösung des neuen Aufgabenblattes bginnen und dabei
Unklarheiten oder andere Schwierigkeiten an die anwesenden HiWis stellen können.
\begin{itemize}
\item {} 
\sphinxAtStartPar
{\hyperref[\detokenize{Notebooks/HelloWorld::doc}]{\sphinxcrossref{Hello World!}}}

\end{itemize}

\sphinxstepscope


\chapter{Hello World!}
\label{\detokenize{Notebooks/HelloWorld:hello-world}}\label{\detokenize{Notebooks/HelloWorld::doc}}
\sphinxAtStartPar
Das “Hello World! \sphinxhyphen{} Programm” ist traditionell das erste Programm, welches man in einer neuen Programmiersprache schreibt.
Aber was ist überhaupt das “Hello World! \sphinxhyphen{} Programm”? Dieses Programm soll tatsächlich nur “Hello World!” ausgeben. Das klingt
sehr simpel (ist es relativ schnell auch), es hilft aber dennoch sich bereits mit grundsätzlichen Strukturen vertraut zu machen.
Deshalb wird dieses besondere Programm auch in diesem Kurs der Startpunkt für Python sein.

\begin{sphinxuseclass}{cell}\begin{sphinxVerbatimInput}

\begin{sphinxuseclass}{cell_input}
\begin{sphinxVerbatim}[commandchars=\\\{\}]
\PYG{c+c1}{\PYGZsh{} Hier beginnt das Hello World! \PYGZhy{} Programm}
\PYG{k}{def} \PYG{n+nf}{main}\PYG{p}{(}\PYG{p}{)}\PYG{p}{:}
    \PYG{c+c1}{\PYGZsh{} wir geben \PYGZdq{}Hello World!\PYGZdq{} aus}
    \PYG{n+nb}{print}\PYG{p}{(}\PYG{l+s+s2}{\PYGZdq{}}\PYG{l+s+s2}{Hello World!}\PYG{l+s+s2}{\PYGZdq{}}\PYG{p}{)}

\PYG{c+c1}{\PYGZsh{} wir starten das Programm}
\PYG{n}{main}\PYG{p}{(}\PYG{p}{)}
\end{sphinxVerbatim}

\end{sphinxuseclass}\end{sphinxVerbatimInput}
\begin{sphinxVerbatimOutput}

\begin{sphinxuseclass}{cell_output}
\begin{sphinxVerbatim}[commandchars=\\\{\}]
Hello World!
\end{sphinxVerbatim}

\end{sphinxuseclass}\end{sphinxVerbatimOutput}

\end{sphinxuseclass}
\sphinxAtStartPar
Diese kleine Programm besteht im wesentlichen aus drei Komponenten.
\begin{enumerate}
\sphinxsetlistlabels{\arabic}{enumi}{enumii}{}{.}%
\item {} 
\sphinxAtStartPar
Einem \sphinxstylestrong{Kommentar:} Hierbei handelt es sich um eine Notiz des Programmieres. Eingeleitet wird dies durch das ‘\#’\sphinxhyphen{}Symbol. Alles was hinter diesem Sybmol geschrieben steht, wird komplett bei der Übersetzung des Programms ignoriert.

\item {} 
\sphinxAtStartPar
Einer \sphinxstylestrong{Funktionendefinition:} Damit man Funktionen nutzen kann, muss man diese vorher definieren. Danach kann man diese beliebig oft benutzen. Man muss noch erwähnen, dass das main()\sphinxhyphen{}Programm ein besonderes Programm ist. Dieses dient im Allgemeinen als Startpunkt für die Übersetzung. Die Syntax ist hier immer die gleiche. Zuerst das Schlüsselwort \sphinxcode{\sphinxupquote{def}}, danach der Name der Funktion, gefolgt von einem Doppelpunkt.

\item {} 
\sphinxAtStartPar
Einem \sphinxstylestrong{Funktionenaufruf:} Streng genommen sind hier zwei Aufrufe. Der erste in Zeile 3 und der zweite in Zeile 5. Der Aufruf an print() schreibt den Inhalt (hier: “Hello World!”) raus. Der Aufruf an main() startet das Programm und gibt dann “Hello World!” aus. Außerdem wird dem aufmerksamen Leser aufgefallen sein, dass wir \sphinxcode{\sphinxupquote{print()}} nicht definiert hatten, obwohl doch eigentlich jede Funktion vorher definert sein müsste. Hier ist es allerdings ein wenig anders. Python liefert direkt die \sphinxcode{\sphinxupquote{print()}}\sphinxhyphen{}Funktion mit aus, und sie damit einfach verfügbar. Es gibt noch weitere solcher Funktionen, die bei gegebener Zeit noch eingeführt werden.

\end{enumerate}

\begin{sphinxadmonition}{note}{Das main()\sphinxhyphen{}Programm in Python}

\sphinxAtStartPar
Es ist nicht notwendig in Python mit dem main() Programm zu starten. Deshalb wird es bald nicht mehr genutzt werden. Da aber hiermit einfach die Syntax für eine Funktionen\sphinxhyphen{}Definition gezeigt werden kann wurde es hier gezeigt. Außerdem wird in den meisten anderen Programmiersprachen das main() Programm als Einstieg verlangt und somit haben Sie ein solches Programm dann schon einmal gesehen. Einige Beispiele wären (nur zur Illustration):

\begin{sphinxuseclass}{sd-tab-set}
\begin{sphinxuseclass}{sd-tab-item}\subsubsection*{C}

\begin{sphinxuseclass}{sd-tab-content}
\begin{sphinxVerbatim}[commandchars=\\\{\}]
\PYG{c+cp}{\PYGZsh{}}\PYG{+w}{ }\PYG{c+cp}{include}\PYG{+w}{ }\PYG{c+cpf}{\PYGZlt{}stdio.h\PYGZgt{}}

\PYG{k+kt}{int}\PYG{+w}{ }\PYG{n+nf}{main}\PYG{p}{(}\PYG{p}{)}\PYG{+w}{ }\PYG{p}{\PYGZob{}}
\PYG{+w}{    }\PYG{n}{printf}\PYG{p}{(}\PYG{l+s}{\PYGZdq{}}\PYG{l+s}{Hello World!}\PYG{l+s+se}{\PYGZbs{}n}\PYG{l+s}{\PYGZdq{}}\PYG{p}{)}\PYG{p}{;}
\PYG{+w}{    }\PYG{k}{return}\PYG{+w}{ }\PYG{l+m+mi}{0}\PYG{p}{;}
\PYG{p}{\PYGZcb{}}
\end{sphinxVerbatim}

\end{sphinxuseclass}
\end{sphinxuseclass}
\begin{sphinxuseclass}{sd-tab-item}\subsubsection*{C++}

\begin{sphinxuseclass}{sd-tab-content}
\begin{sphinxVerbatim}[commandchars=\\\{\}]
\PYG{c+cp}{\PYGZsh{}}\PYG{c+cp}{include}\PYG{+w}{ }\PYG{c+cpf}{\PYGZlt{}iostream\PYGZgt{}}

\PYG{k+kt}{int}\PYG{+w}{ }\PYG{n+nf}{main}\PYG{p}{(}\PYG{p}{)}\PYG{+w}{ }
\PYG{p}{\PYGZob{}}
\PYG{+w}{    }\PYG{n}{std}\PYG{o}{:}\PYG{o}{:}\PYG{n}{cout}\PYG{+w}{ }\PYG{o}{\PYGZlt{}}\PYG{o}{\PYGZlt{}}\PYG{+w}{ }\PYG{l+s}{\PYGZdq{}}\PYG{l+s}{Hello World!}\PYG{l+s}{\PYGZdq{}}\PYG{+w}{ }\PYG{o}{\PYGZlt{}}\PYG{o}{\PYGZlt{}}\PYG{+w}{ }\PYG{n}{std}\PYG{o}{:}\PYG{o}{:}\PYG{n}{endl}\PYG{p}{;}
\PYG{+w}{    }\PYG{k}{return}\PYG{+w}{ }\PYG{l+m+mi}{0}\PYG{p}{;}
\PYG{p}{\PYGZcb{}}
\end{sphinxVerbatim}

\end{sphinxuseclass}
\end{sphinxuseclass}
\begin{sphinxuseclass}{sd-tab-item}\subsubsection*{Java}

\begin{sphinxuseclass}{sd-tab-content}
\begin{sphinxVerbatim}[commandchars=\\\{\}]
\PYG{k+kd}{class} \PYG{n+nc}{Main}\PYG{+w}{ }\PYG{p}{\PYGZob{}}
\PYG{+w}{    }\PYG{k+kd}{public}\PYG{+w}{ }\PYG{k+kd}{static}\PYG{+w}{ }\PYG{k+kt}{void}\PYG{+w}{ }\PYG{n+nf}{main}\PYG{p}{(}\PYG{n}{String}\PYG{o}{[}\PYG{o}{]}\PYG{+w}{ }\PYG{n}{args}\PYG{p}{)}\PYG{+w}{ }\PYG{p}{\PYGZob{}}
\PYG{+w}{        }\PYG{n}{System}\PYG{p}{.}\PYG{n+na}{out}\PYG{p}{.}\PYG{n+na}{println}\PYG{p}{(}\PYG{l+s}{\PYGZdq{}}\PYG{l+s}{Hello World!}\PYG{l+s}{\PYGZdq{}}\PYG{p}{)}\PYG{p}{;}
\PYG{+w}{    }\PYG{p}{\PYGZcb{}}
\PYG{p}{\PYGZcb{}}
\end{sphinxVerbatim}

\end{sphinxuseclass}
\end{sphinxuseclass}
\begin{sphinxuseclass}{sd-tab-item}\subsubsection*{Rust}

\begin{sphinxuseclass}{sd-tab-content}
\begin{sphinxVerbatim}[commandchars=\\\{\}]
\PYG{k}{fn} \PYG{n+nf}{main}\PYG{p}{(}\PYG{p}{)}\PYG{+w}{ }\PYG{p}{\PYGZob{}}
\PYG{+w}{    }\PYG{n+nf+fm}{println!}\PYG{p}{(}\PYG{l+s}{\PYGZdq{}}\PYG{l+s}{Hello World!}\PYG{l+s}{\PYGZdq{}}\PYG{p}{)}\PYG{p}{;}
\end{sphinxVerbatim}

\end{sphinxuseclass}
\end{sphinxuseclass}
\end{sphinxuseclass}\end{sphinxadmonition}


\section{Das Hello World! \sphinxhyphen{} Programm Improved}
\label{\detokenize{Notebooks/HelloWorld:das-hello-world-programm-improved}}
\sphinxAtStartPar
Zum Beispiel betrachten wir folgendes Prgramm:

\begin{sphinxuseclass}{cell}\begin{sphinxVerbatimInput}

\begin{sphinxuseclass}{cell_input}
\begin{sphinxVerbatim}[commandchars=\\\{\}]
\PYG{k}{def} \PYG{n+nf}{main}\PYG{p}{(}\PYG{p}{)}\PYG{p}{:}
    \PYG{n+nb}{print}\PYG{p}{(}\PYG{l+s+s2}{\PYGZdq{}}\PYG{l+s+s2}{Hello World!}\PYG{l+s+s2}{\PYGZdq{}}\PYG{p}{)}
    \PYG{n+nb}{print}\PYG{p}{(}\PYG{l+s+s2}{\PYGZdq{}}\PYG{l+s+s2}{Hello World!}\PYG{l+s+s2}{\PYGZdq{}}\PYG{p}{)}

\PYG{n}{main}\PYG{p}{(}\PYG{p}{)}
\end{sphinxVerbatim}

\end{sphinxuseclass}\end{sphinxVerbatimInput}
\begin{sphinxVerbatimOutput}

\begin{sphinxuseclass}{cell_output}
\begin{sphinxVerbatim}[commandchars=\\\{\}]
Hello World!
Hello World!
\end{sphinxVerbatim}

\end{sphinxuseclass}\end{sphinxVerbatimOutput}

\end{sphinxuseclass}
\sphinxAtStartPar
\sphinxstylestrong{Wir können auch mehrfach Hello World! ausgeben.}

\sphinxAtStartPar
Und sogar noch öfter:

\begin{sphinxuseclass}{cell}\begin{sphinxVerbatimInput}

\begin{sphinxuseclass}{cell_input}
\begin{sphinxVerbatim}[commandchars=\\\{\}]
\PYG{k}{def} \PYG{n+nf}{main}\PYG{p}{(}\PYG{p}{)}\PYG{p}{:}
    \PYG{n+nb}{print}\PYG{p}{(}\PYG{l+s+s2}{\PYGZdq{}}\PYG{l+s+s2}{Hello World!}\PYG{l+s+s2}{\PYGZdq{}}\PYG{p}{)}
    \PYG{n+nb}{print}\PYG{p}{(}\PYG{l+s+s2}{\PYGZdq{}}\PYG{l+s+s2}{Hello World!}\PYG{l+s+s2}{\PYGZdq{}}\PYG{p}{)}
    \PYG{n+nb}{print}\PYG{p}{(}\PYG{l+s+s2}{\PYGZdq{}}\PYG{l+s+s2}{Hello World!}\PYG{l+s+s2}{\PYGZdq{}}\PYG{p}{)}

\PYG{n}{main}\PYG{p}{(}\PYG{p}{)}
\end{sphinxVerbatim}

\end{sphinxuseclass}\end{sphinxVerbatimInput}
\begin{sphinxVerbatimOutput}

\begin{sphinxuseclass}{cell_output}
\begin{sphinxVerbatim}[commandchars=\\\{\}]
Hello World!
Hello World!
Hello World!
\end{sphinxVerbatim}

\end{sphinxuseclass}\end{sphinxVerbatimOutput}

\end{sphinxuseclass}
\sphinxAtStartPar
Aber auf die Dauer wird dies dann doch ganz schön schreibaufwendig. Aber Python (und alle anderen Programmiersprachen) hält hierfür etwas sehr vereinfachendes bereit: \sphinxstyleemphasis{\sphinxstylestrong{Schleifen}}


\subsection{Schleifen (engl. Loops)}
\label{\detokenize{Notebooks/HelloWorld:schleifen-engl-loops}}
\sphinxAtStartPar
Also schreiben wir unser letztes Programm ein wenig um.

\begin{sphinxuseclass}{cell}\begin{sphinxVerbatimInput}

\begin{sphinxuseclass}{cell_input}
\begin{sphinxVerbatim}[commandchars=\\\{\}]
\PYG{k}{def} \PYG{n+nf}{main}\PYG{p}{(}\PYG{p}{)}\PYG{p}{:}
    \PYG{k}{for} \PYG{n}{i} \PYG{o+ow}{in} \PYG{n+nb}{range}\PYG{p}{(}\PYG{l+m+mi}{3}\PYG{p}{)}\PYG{p}{:}
        \PYG{n+nb}{print}\PYG{p}{(}\PYG{l+s+s2}{\PYGZdq{}}\PYG{l+s+s2}{Hello World!}\PYG{l+s+s2}{\PYGZdq{}}\PYG{p}{)}

\PYG{n}{main}\PYG{p}{(}\PYG{p}{)}
\end{sphinxVerbatim}

\end{sphinxuseclass}\end{sphinxVerbatimInput}
\begin{sphinxVerbatimOutput}

\begin{sphinxuseclass}{cell_output}
\begin{sphinxVerbatim}[commandchars=\\\{\}]
Hello World!
Hello World!
Hello World!
\end{sphinxVerbatim}

\end{sphinxuseclass}\end{sphinxVerbatimOutput}

\end{sphinxuseclass}
\sphinxAtStartPar
Dies ist eine sogenannte \sphinxcode{\sphinxupquote{for}}\sphinxhyphen{}Schleife. \sphinxcode{\sphinxupquote{for}}\sphinxhyphen{}Schleifen sind bei weitem die meist genutzten Schleifen in Python.

\sphinxAtStartPar
Die Syntax ist im wesentlichen immer die wie oben gezeigt. Die beiden auffälligen Komponenten werden noch hier einzeln erläutert:
\begin{itemize}
\item {} 
\sphinxAtStartPar
\sphinxcode{\sphinxupquote{range()}}: Diese Funktion gibt eine Reihe oder Folge (engl. range) von (Ganz\sphinxhyphen{})Zahlen zurück. Hierbei gibt es insgsamt drei verschiedene Fälle:
\begin{itemize}
\item {} 
\sphinxAtStartPar
\sphinxcode{\sphinxupquote{range(n)}}: Hier wird eine Reihe von Zahlen beginnend bei 0 und endend bei \sphinxstylestrong{n\sphinxhyphen{}1} (also insgesamt n Zahlen) erzeugt

\item {} 
\sphinxAtStartPar
\sphinxcode{\sphinxupquote{range(a,b)}}: Hier wird eine Reihe von Zahlen beginnend bei \sphinxcode{\sphinxupquote{a}} und endend bei \sphinxstylestrong{b\sphinxhyphen{}1} erzeugt.

\item {} 
\sphinxAtStartPar
\sphinxcode{\sphinxupquote{range(a,b,inc)}}: Hier wird eine Reihe von Zahlen erzeugt, die bei \sphinxcode{\sphinxupquote{a}} startet, wobei immer um \sphinxcode{\sphinxupquote{inc}} hochgezählt wird. Wenn das letzte Ergbnis \(> b-1\) ist, so wird die letzt mögliche Zahl, für die gilt, dass sie kleiner als \sphinxcode{\sphinxupquote{b\sphinxhyphen{}1}} zurückgegeben. In Code sieht dies vereinfacht wie folgt aus:

\end{itemize}

\end{itemize}

\begin{sphinxVerbatim}[commandchars=\\\{\}]
\PYG{k}{def} \PYG{n+nf}{my\PYGZus{}range}\PYG{p}{(}\PYG{n}{start}\PYG{p}{,} \PYG{n}{end}\PYG{p}{,} \PYG{n}{inc}\PYG{p}{)}\PYG{p}{:}
    \PYG{c+c1}{\PYGZsh{} wir müssen uns alle berechneten Zahlen merken. Dies kann man in Python in einer sogenannten }
    \PYG{c+c1}{\PYGZsh{} Liste machen. Später mehr dazu.}
    \PYG{n}{range\PYGZus{}list} \PYG{o}{=} \PYG{n+nb}{list}\PYG{p}{(}\PYG{p}{)}
    
    \PYG{c+c1}{\PYGZsh{} while\PYGZhy{}Schleifen brauchen uns im Moment nicht wirklich zu interessieren. Diese Art Schleife }
    \PYG{c+c1}{\PYGZsh{} führt solange einen bestimmten Programmteil immer und immer wieder aus, bis eine bestimmte }
    \PYG{c+c1}{\PYGZsh{} Bedingung NICHT MEHR erfüllt ist (hier also: sobald start \PYGZgt{} end ist, brechen wir ab)}
    \PYG{k}{while} \PYG{n}{start} \PYG{o}{\PYGZlt{}} \PYG{n}{end}\PYG{p}{:}
        \PYG{n}{range\PYGZus{}list}\PYG{o}{.}\PYG{n}{append}\PYG{p}{(}\PYG{n}{start}\PYG{p}{)}
        \PYG{n}{start} \PYG{o}{+}\PYG{o}{=} \PYG{n}{inc}
        
    \PYG{c+c1}{\PYGZsh{} jetzt geben wir die Liste zurück}
    \PYG{k}{return} \PYG{n}{range\PYGZus{}list}
\end{sphinxVerbatim}

\sphinxAtStartPar
Die Schleife aus dem letzten Code\sphinxhyphen{}Schnipsel ist eine sogenannte \sphinxcode{\sphinxupquote{While}}\sphinxhyphen{}Schleife. \sphinxcode{\sphinxupquote{While}}\sphinxhyphen{}Schleifen führen einen Code\sphinxhyphen{}Abschnitt genauso lange aus, wie eine Bedingung gilt (also in unserem Beispiel solange \sphinxcode{\sphinxupquote{start < end}}).

\sphinxAtStartPar
\sphinxstylestrong{Jedoch Vorsicht mit \sphinxcode{\sphinxupquote{While}}\sphinxhyphen{}Schleifen}. Betrachten wir das folgende kleine Programm:

\begin{sphinxVerbatim}[commandchars=\\\{\}]
\PYG{k}{def} \PYG{n+nf}{my\PYGZus{}never\PYGZus{}ending\PYGZus{}loop}\PYG{p}{(}\PYG{p}{)}\PYG{p}{:}
    \PYG{c+c1}{\PYGZsh{} \PYGZsq{}==\PYGZsq{} steht für den Test auf Gleichheit; wir testen also, ob 1 gleich 1 ist}
    \PYG{k}{while} \PYG{l+m+mi}{1} \PYG{o}{==} \PYG{l+m+mi}{1}\PYG{p}{:}
        \PYG{n}{myvar} \PYG{o}{=} \PYG{l+m+mi}{1}
\end{sphinxVerbatim}

\sphinxAtStartPar
Es sollte ziemlich offensichtlich sein, dass diese Schleife niemals enden wird. Deswegen werden wir in den meisten Fällen eher selten \sphinxcode{\sphinxupquote{While}}\sphinxhyphen{}Schleifen zu Gesicht bekommen und beschränken uns hauptsächlich auf \sphinxcode{\sphinxupquote{for}}\sphinxhyphen{}Schleifen. Nur damit haben Sie schon einmal eine \sphinxcode{\sphinxupquote{While}}\sphinxhyphen{}Schleife gesehen, und sind nicht komplett hilflos, falls Sie eine einmal sehen sollten.


\subsection{Die \sphinxstyleliteralintitle{\sphinxupquote{if\sphinxhyphen{}else}} \sphinxhyphen{} Anweisung}
\label{\detokenize{Notebooks/HelloWorld:die-if-else-anweisung}}
\sphinxAtStartPar
Wir schauen uns noch eine weitere Erweiterung des Hello World! \sphinxhyphen{} Programmes an. Zum Beispiel möchte man an allen Tagen außer Montag \sphinxcode{\sphinxupquote{Hello beautiful World!}} ausgeben, und montags \sphinxcode{\sphinxupquote{hello world}}. Eine mögliche Implementierung wäre die Folgende:

\begin{sphinxuseclass}{cell}\begin{sphinxVerbatimInput}

\begin{sphinxuseclass}{cell_input}
\begin{sphinxVerbatim}[commandchars=\\\{\}]
\PYG{k}{def} \PYG{n+nf}{main}\PYG{p}{(}\PYG{p}{)}\PYG{p}{:}
    \PYG{n}{is\PYGZus{}monday} \PYG{o}{=} \PYG{k+kc}{False}
    
    \PYG{k}{if} \PYG{n}{is\PYGZus{}monday}\PYG{p}{:}
        \PYG{n+nb}{print}\PYG{p}{(}\PYG{l+s+s2}{\PYGZdq{}}\PYG{l+s+s2}{hello world}\PYG{l+s+s2}{\PYGZdq{}}\PYG{p}{)}
    \PYG{k}{else}\PYG{p}{:}
        \PYG{n+nb}{print}\PYG{p}{(}\PYG{l+s+s2}{\PYGZdq{}}\PYG{l+s+s2}{Hello beautiful World!}\PYG{l+s+s2}{\PYGZdq{}}\PYG{p}{)}
        
\PYG{n}{main}\PYG{p}{(}\PYG{p}{)}
\end{sphinxVerbatim}

\end{sphinxuseclass}\end{sphinxVerbatimInput}
\begin{sphinxVerbatimOutput}

\begin{sphinxuseclass}{cell_output}
\begin{sphinxVerbatim}[commandchars=\\\{\}]
Hello beautiful World!
\end{sphinxVerbatim}

\end{sphinxuseclass}\end{sphinxVerbatimOutput}

\end{sphinxuseclass}
\sphinxAtStartPar
Das einzige wirklich neue/nicht direkt sich selbst erklärende Schlüsselwort sollte \sphinxcode{\sphinxupquote{False}} sein. \sphinxcode{\sphinxupquote{False}} ist englisch und steht einfach nur für \sphinxstyleemphasis{Falsch}. Ist also eine Bedinung \sphinxstyleemphasis{Falsch}, so wird in einer if \sphinxhyphen{} Anweisung der \sphinxcode{\sphinxupquote{else}}\sphinxhyphen{}Zweig ausgeführt. Das Gegenstück zu \sphinxcode{\sphinxupquote{False}} ist \sphinxcode{\sphinxupquote{True}} (= wahr). Wäre, um in unserem Beispiel zu bleiben, heute Montag, so bekämen wir \sphinxcode{\sphinxupquote{Hello beautiful World!}} als Ausgabe. Insgesamt wird dies unter dem Begriff der booleschen Algebra zusammengefasst.

\begin{sphinxadmonition}{note}{Boolesche Algebra}

\sphinxAtStartPar
Boolesche Algebra steht im Übrigen auch hinter der Bedingung in der \sphinxcode{\sphinxupquote{While}}\sphinxhyphen{}Schleife! Denn solange die Bedingung der Schleife wahr ist (also zu \sphinxcode{\sphinxupquote{True}} auswertet), wird der Schleifenrumpf ausgeführt. Sobald die Bedingung nicht mehr wahr ist (also zu \sphinxcode{\sphinxupquote{False}} auswertet), wird die Schleife abgebrochen, und das restliche Programm nach der Schleife (welches auch leer sein kann) wird ausgeführt.
\end{sphinxadmonition}


\bigskip\hrule\bigskip


\sphinxAtStartPar
\sphinxstyleemphasis{\sphinxstylestrong{Damit ist unsere grundlegende Einleitung in Python abgeschlossen!}} In den nächsten Kapiteln werden die hier erlernten Fähigkeiten dann für Probleme mit physikalischem Hintergrund genutzt werden. Ebenso werden wir ab jetzt das eigentlich unnötige \sphinxcode{\sphinxupquote{main()}}\sphinxhyphen{}Programm in Python nicht mehr benutzen und es darf damit wegelassen werden.







\renewcommand{\indexname}{Index}
\printindex
\end{document}