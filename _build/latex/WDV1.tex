%% Generated by Sphinx.
\def\sphinxdocclass{jupyterBook}
\documentclass[letterpaper,10pt,english]{jupyterBook}
\ifdefined\pdfpxdimen
   \let\sphinxpxdimen\pdfpxdimen\else\newdimen\sphinxpxdimen
\fi \sphinxpxdimen=.75bp\relax
\ifdefined\pdfimageresolution
    \pdfimageresolution= \numexpr \dimexpr1in\relax/\sphinxpxdimen\relax
\fi
%% let collapsible pdf bookmarks panel have high depth per default
\PassOptionsToPackage{bookmarksdepth=5}{hyperref}
%% turn off hyperref patch of \index as sphinx.xdy xindy module takes care of
%% suitable \hyperpage mark-up, working around hyperref-xindy incompatibility
\PassOptionsToPackage{hyperindex=false}{hyperref}
%% memoir class requires extra handling
\makeatletter\@ifclassloaded{memoir}
{\ifdefined\memhyperindexfalse\memhyperindexfalse\fi}{}\makeatother

\PassOptionsToPackage{warn}{textcomp}

\catcode`^^^^00a0\active\protected\def^^^^00a0{\leavevmode\nobreak\ }
\usepackage{cmap}
\usepackage{fontspec}
\defaultfontfeatures[\rmfamily,\sffamily,\ttfamily]{}
\usepackage{amsmath,amssymb,amstext}
\usepackage{polyglossia}
\setmainlanguage{english}



\setmainfont{FreeSerif}[
  Extension      = .otf,
  UprightFont    = *,
  ItalicFont     = *Italic,
  BoldFont       = *Bold,
  BoldItalicFont = *BoldItalic
]
\setsansfont{FreeSans}[
  Extension      = .otf,
  UprightFont    = *,
  ItalicFont     = *Oblique,
  BoldFont       = *Bold,
  BoldItalicFont = *BoldOblique,
]
\setmonofont{FreeMono}[
  Extension      = .otf,
  UprightFont    = *,
  ItalicFont     = *Oblique,
  BoldFont       = *Bold,
  BoldItalicFont = *BoldOblique,
]



\usepackage[Bjarne]{fncychap}
\usepackage[,numfigreset=1,mathnumfig]{sphinx}

\fvset{fontsize=\small}
\usepackage{geometry}


% Include hyperref last.
\usepackage{hyperref}
% Fix anchor placement for figures with captions.
\usepackage{hypcap}% it must be loaded after hyperref.
% Set up styles of URL: it should be placed after hyperref.
\urlstyle{same}


\usepackage{sphinxmessages}



        % Start of preamble defined in sphinx-jupyterbook-latex %
         \usepackage[Latin,Greek]{ucharclasses}
        \usepackage{unicode-math}
        % fixing title of the toc
        \addto\captionsenglish{\renewcommand{\contentsname}{Contents}}
        \hypersetup{
            pdfencoding=auto,
            psdextra
        }
        % End of preamble defined in sphinx-jupyterbook-latex %
        

\title{WDV1}
\date{Apr 12, 2023}
\release{}
\author{Marc Johannes Honecker}
\newcommand{\sphinxlogo}{\vbox{}}
\renewcommand{\releasename}{}
\makeindex
\begin{document}

\pagestyle{empty}
\sphinxmaketitle
\pagestyle{plain}
\sphinxtableofcontents
\pagestyle{normal}
\phantomsection\label{\detokenize{intro::doc}}


\sphinxAtStartPar
In diesem Kurs soll Ihnen beigebracht werden, Computer zur Unterstützung diverser anderer Fächer zu verwenden. Als die dazu beste
Programmiersprache hat sich (momentan) Python herauskristallisiert, auf das sich dieser Kurs fokussiert. Python ist bereits seit
einigen Jahren State\sphinxhyphen{}Of\sphinxhyphen{}The\sphinxhyphen{}Art für viele Bereiche, sei es Scientific Computing oder (vor allem in den letzten Jahren) KI.
Deshalb ist im Moment ein Ende seines Siegeszuges noch nicht absehbar.

\sphinxAtStartPar
Für den Kurs haben wir einen “ganzheitlichen” Ansatz gewählt, der von üblichen Programmierkursen abweicht. Sie lernen also im
übertragenen Sinne zunächst ganze Sätze ohne mit vielen grammatikalischen Begriffen konfrontiert zu werden. Damit sollen Sie
schnell in die Lage versetzt werden kontinuierlich durch das Skript an Funktionen verschiedener Bibliotheken herangefürht zu
werden. Damit werdne an einigen Beispielen auch Hitergrund z.B. zu Numerik oder Statistik formlos gelehrt oder es wird aufgezeigt,
wie die Probleme mit herkömmlichen Programmen, allerdings in Python Syntax gelöst würden. Am Ende von WDV\sphinxhyphen{}1 sollten Sie dann in
der Lage sein, echte Python Bücher (quer) zu lesen, aber auch schnell andere Programmiersprachen wie z.B. C++ zu erlernen.

\sphinxAtStartPar
In den Hausaufgaben wird es zwei Testate geben, deren Bestehen Ihnen erlaubt an der finalen Leistungkontrolle, der Klausur,
teilzunehmen. Weder bei Testaten, noch Klausuren sind Computer erlaubt. Das klingt paradox, aber es wäre sonst zu einfach zu
mogeln. Leistungkontrollen werden vermutlich aus zwei, manchmal drei Teilen bestehen:
\begin{enumerate}
\sphinxsetlistlabels{\arabic}{enumi}{enumii}{}{.}%
\item {} 
\sphinxAtStartPar
Ein Teil, in dem Ihnen der Code gegeben wird und Sie vorhersagen müssen, was bei dei dessen Ausführung passiert.

\item {} 
\sphinxAtStartPar
Ein Teil, in dem Sie selbst Python Code schreiben.

\item {} 
\sphinxAtStartPar
Unter Umständen werden Sie noch aufgefordert, Fehler in einem gegebenen Code zu markieren/verbessern.

\end{enumerate}

\sphinxAtStartPar
\sphinxstylestrong{Dieses Skript ist kein Lehrbuch.} Details zur Syntax und andere Regeln werdne in der Vorlesung und den Übungen erklärt oder
können in den angegebenen Referenzen nachgeschlagen werden. Dieses Skript ist eher eine Ansammlung von kleinen kommentierten
Code\sphinxhyphen{}Schnipsel.

\begin{DUlineblock}{0em}
\item[] \sphinxstylestrong{\Large Ablauf einer Lehreinheit}
\end{DUlineblock}

\sphinxAtStartPar
Eine Lehreinheit besteht aus 3 Lehrstunden zu je 45 Minuten. In der 1. Stunde werden die Hausaufgaben besprochen, in der 2. Stunde
neue Inhalte vermittelt und die 3. Stunde dient dazu, dass Studierende mit der Lösung des neuen Aufgabenblattes bginnen und dabei
Unklarheiten oder andere Schwierigkeiten an die anwesenden HiWis stellen können.
\begin{itemize}
\item {} 
\sphinxAtStartPar
{\hyperref[\detokenize{Notebooks/Hello World!::doc}]{\sphinxcrossref{Hello World!}}}

\end{itemize}

\sphinxstepscope


\chapter{Hello World!}
\label{\detokenize{Notebooks/Hello World!:hello-world}}\label{\detokenize{Notebooks/Hello World!::doc}}
\sphinxAtStartPar
Das “Hello World! \sphinxhyphen{} Programm” ist traditionell das erste Programm, welches man in einer neuen Programmiersprache schreibt.
Aber was ist überhaupt das “Hello World! \sphinxhyphen{} Programm”? Dieses Programm soll tatsächlich nur “Hello World!” ausgeben. Das klingt
sehr simpel (ist es relativ schnell auch), es hilft aber dennoch sich bereits mit grundsätzlichen Strukturen vertraut zu machen.
Deshalb wird dieses besondere Programm auch in diesem Kurs der Startpunkt für Python sein.

\begin{sphinxuseclass}{cell}\begin{sphinxVerbatimInput}

\begin{sphinxuseclass}{cell_input}
\begin{sphinxVerbatim}[commandchars=\\\{\}]
\PYG{c+c1}{\PYGZsh{} Hier beginnt das Hello World! \PYGZhy{} Programm}
\PYG{k}{def} \PYG{n+nf}{main}\PYG{p}{(}\PYG{p}{)}\PYG{p}{:}
    \PYG{n+nb}{print}\PYG{p}{(}\PYG{l+s+s2}{\PYGZdq{}}\PYG{l+s+s2}{Hello World!}\PYG{l+s+s2}{\PYGZdq{}}\PYG{p}{)}
    
\PYG{n}{main}\PYG{p}{(}\PYG{p}{)}
\end{sphinxVerbatim}

\end{sphinxuseclass}\end{sphinxVerbatimInput}
\begin{sphinxVerbatimOutput}

\begin{sphinxuseclass}{cell_output}
\begin{sphinxVerbatim}[commandchars=\\\{\}]
Hello World!
\end{sphinxVerbatim}

\end{sphinxuseclass}\end{sphinxVerbatimOutput}

\end{sphinxuseclass}
\sphinxAtStartPar
Diese kleine Programm besteht im wesentlichen aus drei Komponenten.
\begin{enumerate}
\sphinxsetlistlabels{\arabic}{enumi}{enumii}{}{.}%
\item {} 
\sphinxAtStartPar
Einem \sphinxstylestrong{Kommentar:} Hierbei handelt es sich um eine Notiz des Programmieres. Eingeleitet wird dies durch das ‘\#’\sphinxhyphen{}Symbol. Alles was hinter diesem Sybmol geschrieben steht, wird komplett bei der Übersetzung des Programms ignoriert.

\item {} 
\sphinxAtStartPar
Einer \sphinxstylestrong{Funktionendefinition:} Damit man Funktionen nutzen kann, muss man diese vorher definieren. Danach kann man diese beliebig oft benutzen. Man muss noch erwähnen, dass das main()\sphinxhyphen{}Programm ein besonderes Programm ist. Dieses dient im Allgemeinen als Startpunkt für die Übersetzung.

\item {} 
\sphinxAtStartPar
Einem \sphinxstylestrong{Funktionenaufruf:} Streng genommen sind hier zwei Aufrufe. Der erste in Zeile 3 und der zweite in Zeile 5. Der Aufruf an print() schreibt den Inhalt (hier: “Hello World!”) raus. Der Aufruf an main() startet das Programm und gibt dann “Hello World!” aus.

\end{enumerate}
\subsubsection*{Das main()\sphinxhyphen{}Programm in Python}

\sphinxAtStartPar
Es ist nicht notwendig in Python mit dem main() Programm zu starten. Aber in den meisten anderen Programmiersprachen wird das main() Programm als Einstieg verlangt. Einige Beispiele wären (nur zur Illustration):

\begin{sphinxuseclass}{sd-tab-set}
\begin{sphinxuseclass}{sd-tab-item}\subsubsection*{C}

\begin{sphinxuseclass}{sd-tab-content}
\begin{sphinxVerbatim}[commandchars=\\\{\}]
\PYG{c+cp}{\PYGZsh{}}\PYG{+w}{ }\PYG{c+cp}{include}\PYG{+w}{ }\PYG{c+cpf}{\PYGZlt{}stdio.h\PYGZgt{}}

\PYG{k+kt}{int}\PYG{+w}{ }\PYG{n+nf}{main}\PYG{p}{(}\PYG{p}{)}\PYG{+w}{ }\PYG{p}{\PYGZob{}}
\PYG{+w}{    }\PYG{n}{printf}\PYG{p}{(}\PYG{l+s}{\PYGZdq{}}\PYG{l+s}{Hello World!}\PYG{l+s+se}{\PYGZbs{}n}\PYG{l+s}{\PYGZdq{}}\PYG{p}{)}\PYG{p}{;}
\PYG{+w}{    }\PYG{k}{return}\PYG{+w}{ }\PYG{l+m+mi}{0}\PYG{p}{;}
\PYG{p}{\PYGZcb{}}
\end{sphinxVerbatim}

\end{sphinxuseclass}
\end{sphinxuseclass}
\begin{sphinxuseclass}{sd-tab-item}\subsubsection*{C++}

\begin{sphinxuseclass}{sd-tab-content}
\begin{sphinxVerbatim}[commandchars=\\\{\}]
\PYG{c+cp}{\PYGZsh{}}\PYG{c+cp}{include}\PYG{+w}{ }\PYG{c+cpf}{\PYGZlt{}iostream\PYGZgt{}}

\PYG{k+kt}{int}\PYG{+w}{ }\PYG{n+nf}{main}\PYG{p}{(}\PYG{p}{)}\PYG{+w}{ }
\PYG{p}{\PYGZob{}}
\PYG{+w}{    }\PYG{n}{std}\PYG{o}{:}\PYG{o}{:}\PYG{n}{cout}\PYG{+w}{ }\PYG{o}{\PYGZlt{}}\PYG{o}{\PYGZlt{}}\PYG{+w}{ }\PYG{l+s}{\PYGZdq{}}\PYG{l+s}{Hello World!}\PYG{l+s}{\PYGZdq{}}\PYG{+w}{ }\PYG{o}{\PYGZlt{}}\PYG{o}{\PYGZlt{}}\PYG{+w}{ }\PYG{n}{std}\PYG{o}{:}\PYG{o}{:}\PYG{n}{endl}\PYG{p}{;}
\PYG{+w}{    }\PYG{k}{return}\PYG{+w}{ }\PYG{l+m+mi}{0}\PYG{p}{;}
\PYG{p}{\PYGZcb{}}
\end{sphinxVerbatim}

\end{sphinxuseclass}
\end{sphinxuseclass}
\begin{sphinxuseclass}{sd-tab-item}\subsubsection*{Java}

\begin{sphinxuseclass}{sd-tab-content}
\begin{sphinxVerbatim}[commandchars=\\\{\}]
\PYG{k+kd}{class} \PYG{n+nc}{Main}\PYG{+w}{ }\PYG{p}{\PYGZob{}}
\PYG{+w}{    }\PYG{k+kd}{public}\PYG{+w}{ }\PYG{k+kd}{static}\PYG{+w}{ }\PYG{k+kt}{void}\PYG{+w}{ }\PYG{n+nf}{main}\PYG{p}{(}\PYG{n}{String}\PYG{o}{[}\PYG{o}{]}\PYG{+w}{ }\PYG{n}{args}\PYG{p}{)}\PYG{+w}{ }\PYG{p}{\PYGZob{}}
\PYG{+w}{        }\PYG{n}{System}\PYG{p}{.}\PYG{n+na}{out}\PYG{p}{.}\PYG{n+na}{println}\PYG{p}{(}\PYG{l+s}{\PYGZdq{}}\PYG{l+s}{Hello World!}\PYG{l+s}{\PYGZdq{}}\PYG{p}{)}\PYG{p}{;}
\PYG{+w}{    }\PYG{p}{\PYGZcb{}}
\PYG{p}{\PYGZcb{}}
\end{sphinxVerbatim}

\end{sphinxuseclass}
\end{sphinxuseclass}
\begin{sphinxuseclass}{sd-tab-item}\subsubsection*{Rust}

\begin{sphinxuseclass}{sd-tab-content}
\begin{sphinxVerbatim}[commandchars=\\\{\}]
\PYG{k}{fn} \PYG{n+nf}{main}\PYG{p}{(}\PYG{p}{)}\PYG{+w}{ }\PYG{p}{\PYGZob{}}
\PYG{+w}{    }\PYG{n+nf+fm}{println!}\PYG{p}{(}\PYG{l+s}{\PYGZdq{}}\PYG{l+s}{Hello World!}\PYG{l+s}{\PYGZdq{}}\PYG{p}{)}\PYG{p}{;}
\end{sphinxVerbatim}

\end{sphinxuseclass}
\end{sphinxuseclass}
\end{sphinxuseclass}






\renewcommand{\indexname}{Index}
\printindex
\end{document}